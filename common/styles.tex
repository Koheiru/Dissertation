%%% Кодировки и шрифты %%%
\ifxetexorluatex
    \setmainlanguage[babelshorthands=true]{russian}  % Язык по-умолчанию русский с поддержкой приятных команд пакета babel
    \setotherlanguage{english}                       % Дополнительный язык = английский (в американской вариации по-умолчанию)
    \setmonofont{Courier New}
    \newfontfamily\cyrillicfonttt{Courier New}
    \ifXeTeX
        \defaultfontfeatures{Ligatures=TeX,Mapping=tex-text}
    \else
        \defaultfontfeatures{Ligatures=TeX}
    \fi
    \setmainfont{Times New Roman}
    \newfontfamily\cyrillicfont{Times New Roman}
    \setsansfont{Arial}
    \newfontfamily\cyrillicfontsf{Arial}
\else
    \IfFileExists{pscyr.sty}{\renewcommand{\rmdefault}{ftm}}{}
\fi

%%% Подписи %%%
\captionsetup{%
    singlelinecheck=off,                % Многострочные подписи, например у таблиц
    skip=2pt,                           % Вертикальная отбивка между подписью и содержимым рисунка или таблицы определяется ключом
    justification=centering,            % Центрирование подписей, заданных командой \caption
}

%%% Рисунки %%%
\DeclareCaptionLabelSeparator*{emdash}{~--- }             % (ГОСТ 2.105, 4.3.1)
\captionsetup[figure]{labelsep=emdash,position=bottom}

%%% Таблицы %%%
\iffalse
\ifthenelse{\thetabcap = 0}{%
    \newcommand{\tabcapalign}{\raggedright}  % по левому краю страницы или аналога parbox
}{}

\ifthenelse{\thetablaba = 0 \AND \thetabcap = 1}{%
    \newcommand{\tabcapalign}{\raggedright}  % по левому краю страницы или аналога parbox
}{}

\ifthenelse{\thetablaba = 1 \AND \thetabcap = 1}{%
    \newcommand{\tabcapalign}{\centering}    % по центру страницы или аналога parbox
}{}

\ifthenelse{\thetablaba = 2 \AND \thetabcap = 1}{%
    \newcommand{\tabcapalign}{\raggedleft}   % по правому краю страницы или аналога parbox
}{}

\ifthenelse{\thetabtita = 0 \AND \thetabcap = 1}{%
    \newcommand{\tabtitalign}{\raggedright}  % по левому краю страницы или аналога parbox
}{}

\ifthenelse{\thetabtita = 1 \AND \thetabcap = 1}{%
    \newcommand{\tabtitalign}{\centering}    % по центру страницы или аналога parbox
}{}

\ifthenelse{\thetabtita = 2 \AND \thetabcap = 1}{%
    \newcommand{\tabtitalign}{\raggedleft}   % по правому краю страницы или аналога parbox
}{}

\DeclareCaptionFormat{tablenocaption}{\tabcapalign #1\strut}        % Наименование таблицы отсутствует
\ifthenelse{\thetabcap = 0}{%
    \DeclareCaptionFormat{tablecaption}{\tabcapalign #1#2#3}
    \captionsetup[table]{labelsep=emdash}                       % тире как разделитель идентификатора с номером от наименования
}{%
    \DeclareCaptionFormat{tablecaption}{\tabcapalign #1#2\par%  % Идентификатор таблицы на отдельной строке
        \tabtitalign{#3}}                                       % Наименование таблицы строкой ниже
    \captionsetup[table]{labelsep=space}                        % пробельный разделитель идентификатора с номером от наименования
}
\captionsetup[table]{format=tablecaption,singlelinecheck=off,position=top,skip=0pt}  % многострочные наименования и прочее
\DeclareCaptionLabelFormat{continued}{Продолжение таблицы~#2}

\fi

%%% Подписи подрисунков %%%
\renewcommand{\thesubfigure}{\asbuk{subfigure}}           % Буквенные номера подрисунков
\captionsetup[subfigure]{font={normalsize},               % Шрифт подписи названий подрисунков (не отличается от основного)
    labelformat=brace,                                    % Формат обозначения подрисунка
    justification=centering,                              % Выключка подписей (форматирование), один из вариантов            
}
%\DeclareCaptionFont{font12pt}{\fontsize{12pt}{13pt}\selectfont} % объявляем шрифт 12pt для использования в подписях, тут же надо интерлиньяж объявлять, если не наследуется
%\captionsetup[subfigure]{font={font12pt}}                 % Шрифт подписи названий подрисунков (всегда 12pt)

%%% Настройки гиперссылок %%%
\ifLuaTeX
    \hypersetup{
        unicode,                % Unicode encoded PDF strings
    }
\fi

%%% Шаблон %%%
\DeclareRobustCommand{\todo}{\textcolor{red}}       % решаем проблему превращения названия цвета в результате \MakeUppercase, http://tex.stackexchange.com/a/187930/79756 , \DeclareRobustCommand protects \todo from expanding inside \MakeUppercase
\AtBeginDocument{%
    \setlength{\parindent}{2.5em}                   % Абзацный отступ. Должен быть одинаковым по всему тексту и равен пяти знакам (ГОСТ Р 7.0.11-2011, 5.3.7).
}

%%% Списки %%%
% Используем короткое тире (endash) для ненумерованных списков (ГОСТ 2.105-95, пункт 4.1.7, требует дефиса, но так лучше смотрится)
\renewcommand{\labelitemi}{\normalfont\bfseries{--}}

% Перечисление строчными буквами латинского алфавита (ГОСТ 2.105-95, 4.1.7)
%\renewcommand{\theenumi}{\alph{enumi}}
%\renewcommand{\labelenumi}{\theenumi)} 

% Перечисление строчными буквами русского алфавита (ГОСТ 2.105-95, 4.1.7)
%\makeatletter
%\AddEnumerateCounter{\asbuk}{\russian@alph}{щ}      % Управляем списками/перечислениями через пакет enumitem, а он 'не знает' про asbuk, потому 'учим' его
%\makeatother
%\renewcommand{\theenumi}{\asbuk{enumi}}
%\renewcommand{\labelenumi}{\theenumi)} 

\setlist{nosep,%                                    % Единый стиль для всех списков (пакет enumitem), без дополнительных интервалов.
    labelindent=\parindent,leftmargin=*%            % Каждый пункт, подпункт и перечисление записывают с абзацного отступа (ГОСТ 2.105-95, 4.1.8)
}

%%% Переопределение именований, чтобы можно было и в преамбуле использовать %%%
\renewcommand{\chaptername}{Глава}
\renewcommand{\appendixname}{Приложение} % (ГОСТ Р 7.0.11-2011, 5.7)

%%% Определения, теоремы и т.п. %%%
\newtheoremstyle{normal}
    {0pt} % space above
    {0pt} % space below
    {} % body font
    {\parindent} % indent amount
    {\itshape\bfseries} % header font
    {.} % punctuation after header
    {.5em} % space after header
    {} % manually specify header (set empty for normal behaviour)

\theoremstyle{normal}
\newtheorem{Definition}{Определение}    % определения с нумерацией
\newtheorem*{Definition*}{Определение}  % определения без нумерации

\theoremstyle{normal}
\newtheorem{Statement}{Утверждение}    % утверждение с нумерацией
\newtheorem*{Statement*}{Утверждение}  % утверждение без нумерации


