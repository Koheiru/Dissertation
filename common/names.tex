%% Новые переменные, которые могут использоваться во всём проекте
%% ГОСТ 7.0.11-2011
%% 9.2 Оформление текста автореферата диссертации
%% 9.2.1 Общая характеристика работы включает в себя следующие основные структурные
%% элементы:
%% актуальность темы исследования;
%\newcommand{\actualityTXT}{Актуальность темы.}
%% степень ее разработанности;
%\newcommand{\progressTXT}{Степень разработанности темы.}
%% цели и задачи;
%\newcommand{\aimTXT}{Целью}
%\newcommand{\tasksTXT}{задачи}
%% научную новизну;
%\newcommand{\noveltyTXT}{Научная новизна:}
%% теоретическую и практическую значимость работы;
%%\newcommand{\influenceTXT}{Теоретическая и практическая значимость}
%% или чаще используют просто
%\newcommand{\influenceTXT}{Практическая значимость}
%% методологию и методы исследования;
%\newcommand{\methodsTXT}{Mетодология и методы исследования.}
%% положения, выносимые на защиту;
%\newcommand{\defpositionsTXT}{Основные положения, выносимые на~защиту:}
%% степень достоверности и апробацию результатов.
%\newcommand{\reliabilityTXT}{Достоверность}
%\newcommand{\probationTXT}{Апробация работы.}
%
%\newcommand{\contributionTXT}{Личный вклад.}
%\newcommand{\publicationsTXT}{Публикации.}
%
\newcommand{\authorbibtitle}{Публикации автора по теме диссертации}
\newcommand{\fullbibtitle}{Список литературы} % (ГОСТ Р 7.0.11-2011, 4)
\newcommand{\fullglossarytitle}{Список сокращений и условных обозначений} % (ГОСТ Р 7.0.11-2011, 5.4)

%%% Стилистические сокращения %%%
\newcommand{\socalled}{т.н.\xspace}
\newcommand{\other}{др.\xspace}
\newcommand{\ie}{т.е.\xspace}
\newcommand{\etc}{т.п.\xspace}
\newcommand{\inquotes}[1]{<<#1>>}
\newcommand{\onfigure}{на рис.}
\newcommand{\seefigure}{см. рис.}
\newcommand{\inappendix}{в Приложении}
\newcommand{\seeappendix}{см. Приложение}

%\newcommand{\onfigure}[1]{на \ifthenelse{\isempty{#1}}{рис.}{рис.~\ref{#1}}\xspace}
%\newcommand{\seefigure}[1]{см. \ifthenelse{\isempty{#1}}{рис.}{рис. \ref{#1}}\xspace}
%\newcommand{\inappendix}[1]{в \ifthenelse{\isempty{#1}}{Приложении}{Приложении~\ref{#1}}\xspace}
%\newcommand{\seeappendix}[1]{см. \ifthenelse{\isempty{#1}}{Приложение}{Приложение~\ref{#1}}\xspace}

%%% Математические обозначения %%%
\newcommand{\scalar}[1]{\mathit{#1}}
\renewcommand{\vector}[1]{\mathrm{\mathbf{#1}}}
\renewcommand{\matrix}[1]{\mathrm{\mathbf{#1}}}
\newcommand{\vectoritem}[2]{\mathit{\lowercase{#1}}_{#2}}
\newcommand{\matrixitem}[2]{\mathit{\lowercase{#1}}_{#2}}
\newcommand{\usualset}[1]{#1}
\newcommand{\specialset}[1]{\mathbb{#1}}
\newcommand{\customset}[1]{\mathcal{#1}}
\newcommand{\expectation}[1]{\mathbf{E}\left[#1\right]}

%%% Другое... %%%
\newcommand{\IncludeFigure}[3][ht]{%
    \begin{figure}[#1]
        \makebox[\textwidth][c]{\includegraphics[width=1.0\textwidth]{#2}}
        \caption{#3}
        \label{fig:#2}
    \end{figure}
}

\newcommand{\IncludeCenteredFigure}[2]{%
    \vspace*{\fill}
    {
        \centering
        \begin{figure}[h]
            \makebox[\textwidth][c]{\includegraphics[width=1.0\textwidth]{#1}}
            \caption{#2}
            \label{fig:#1}
        \end{figure}
    }
    \vspace*{\fill}
    \clearpage
}