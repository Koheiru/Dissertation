\chapter*{Введение} \label{introduction}
\addcontentsline{toc}{chapter}{Введение}

\textbf{Актуальность работы.} Одна из важнейших научно-технических проблем авиационно-космической отрасли заключается в разработке автономных беспилотных летательных аппаратов (\acr{UAV}). Человек-оператор всё ещё занимает центральную и незаменимую роль в контуре управления современных \acr{UAV}, обеспечивая, помимо функций общего контроля, также функции целеполагания, выбора стратегии поведения, оперативного реагирования на изменяющиеся условия функционирования и \other Однако использование человека-оператора имеет ряд недостатков, начиная с наличия психо-физиологических ограничений человека по скорости и объёму воспринимаемой информации и заканчивая потенциальными проблемами с каналами передачи данных, что в результате увеличивает риски и снижает эффективность управления. При этом его исключение из контура управления затруднено, т.к. формализация и математическое описание поведения таких сложных объектов как \acr{UAV} является крайне сложной задачей, особенно в условиях изменчивой внешней среды, потенциального изменения свойств самого объекта управления и других неопределённостей.

В свете данной проблемы актуальна задача создания \socalled интеллектуальных автономных систем (\acr{IAS})~\cite{Tiumentsev2002-KII}, характерной чертой которых является высокий уровень автономности, \ie самодостаточность в решении заданных прикладных задач в полном объёме без активного участия человека-оператора. Реализация подобной системы требует рассмотрения и решения целого ряда проблем, для многих из которых только сейчас намечаются пути решения. В рамках данной работы рассматривается задача контекстно-зависимого распознавания образов как один из базовых механизмов интеллектуальной системы, который обеспечил бы ей способность накапливать опыт и применять его даже в ситуациях, не предусмотренных изначально. Это позволило бы получить качественно более гибкую и адаптивную систему, и как следствие, более автономную.

Среди подходов к реализации интеллектуальных автономных систем, а также решения рассматриваемой частной задачи, стоит выделить нейросетевой подход. С одной стороны, он предоставляет методологию моделирования самых разных типов объектов и явлений зачастую без необходимости задания их точных аналитических моделей, вместо этого делая акцент на способах извлечения и представления необходимых для успешной работы знаний, получаемых эмпирически. При этом различные виды нейронных сетей (\acr{NN}) могут быть использованы в роли универсальных аппроксиматоров~\cite{Cybenko1989}, систем кластеризации, систем идентификации и \etc~\cite{Haykin2008} С другой стороны, формально данный подход является многопараметрической задачей нелинейной оптимизации, что тесно связывает его с такими областями как регрессионный анализ, байесовская классификация, теория вероятностей, теория оптимизации и \other~--- при этом случается, что нейросетевой модели можно поставить в соответствие какой-либо строгий аналитический метод и разница между ними заключается лишь в форме представления и анализа задачи~\cite{Vorontsov2015-SHAD}. Однако нейросетевой подход обладает тем преимуществом, что\todo{\ldots универсальность\ldots масштабируемость\ldots}

\todo{Краткий обзор и что предлагается\ldots}

\textbf{Объектом исследования} в работе является интеллектуальная автономная система как управляющий контур многомерных нелинейных динамических систем, в частности таких объектов как \acr{UAV}, для которых осложнено или невозможно в принципе построение всеобъемлющей формализованной математической модели.

\textbf{Предметом исследования} в работе являются нейросетевые методы и алгоритмы контекстно-зависимого распознавания образов как механизм накопления и применения опыта в интеллектуальной автономной системе на основе эмпирических данных.

\textbf{Цель работы} заключается в разработке и исследовании нейросетевой модели контекстно-зависимого распознавания образов, обладающей свойствами непрерывного обучения и стабильно-пластичной интеграции новых образов. Для достижения указанной цели были поставлены и решены следующие задачи:
\begin{enumerate}[label=\arabic*)]
    \item Проанализировать существующие методы\todo{...}
    \item Найти подходы к решению\todo{...}
    \item Разработать и исследовать нейросетевую архитектуру\todo{...}
    \item Разработать и исследовать обучающее правило\todo{...}
    \item Разработать и промоделировать программную реализацию\todo{...}
    \item Провести апробацию и оценку программного комплекса\todo{...}
\end{enumerate}

\textbf{Методы исследования}, использованные в ходе решения поставленных задач, связаны с такими областями как теория нейронных сетей, анализ нелинейных дифференциальных уравнений, теория оптимизации, математическая статистика и численные методы.

\textbf{Научная новизна.} \todo{...} % пункты научной новизны с привязкой к соответствию паспорту специальности

\textbf{Положения, выносимые на защиту.} \todo{...}

\textbf{Достоверность и обоснованность результатов}, полученных в ходе выполнения работы, подтверждается корректностью использования математичесого аппарата, теоретическими выкладками и непротиворечивыми результатами вычислительных экспериментов.

\textbf{Положения, выносимые на защиту.} \todo{...}

\textbf{Апробация работы.} Основные результаты диссертации докладывались и обсуждались на следующих научных конференциях и семинарах:
\begin{itemize}[label=$\bullet$]
    \item Международная конференция <<Авиация и космонавтика>> (Москва, МАИ~(НИУ), 2013, 2014, 2015).
    \item Научно-практическая конференция <<Инновации в авиации и космонавтике>> (Москва, МАИ~(НИУ), 2014, 2015).
    \item Всероссийская научная конференция <<Нейрокомпьютеры и их применение>> (Москва, МГППУ, 2014, 2015, 2016).
    \item Всероссийская научно-техническая конференция <<Нейроинформатика>> (Москва, МИФИ~(НИЯУ), 2015).
    \item Международная научная конференция студентов, аспирантов и молодых учёных <<Ломоносов>> (Москва, МГУ им.~Ломоносова, 2015).
    \item Междисциплинарный научный семинар <<Экобионика>> (Москва, МГТУ им.~Баумана, 2015).
\end{itemize}

\textbf{Публикации результатов.} Основные результаты диссертации опубликованы в 14 печатных изданиях, из них 2 статьи в периодических научных изданиях, рекомендованных ВАК РФ, 1 статья в периодическом международном журнале, представленном в базе цитирования Scopus, 1 статья в периодическом научном журнале и 10 работ в трудах и тезисах конференций. Получено свидетельство о государственной регистрации программы для ЭВМ (Приложение~\ref{appendix:programm_registration}). Результаты работы также отражены в отчётах по НИР.

\textbf{Личный вклад автора.} Содержание диссертации и основные положения, выносимые на защиту, отражают персональный вклад автора в опубликованные работы. В совместных работах автору пренадлежат обзоры научных трудов по нейросетевым методам решения задачи распознавания образов, предлагаемые архитектуры нейросетевых структур и алгоритмы их обучения, а так же схемы проведения и результаты вычислительных экспериментов.

\textbf{Объем и структура работы.} Диссертация состоит из~введения, четырёх глав и заключения. Полный объём диссертации составляет \formbytotal{TotPages}{страниц}{у}{ы}{}, включая \formbytotal{totalcount@figure}{рисун}{ок}{ка}{ков} и \formbytotal{totalcount@table}{таблиц}{у}{ы}{}. Список литературы содержит \formbytotal{citenum}{наименован}{ие}{ия}{ий}.
